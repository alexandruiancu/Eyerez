

The heart of the Eyerez system are the two laser distancers which are
actuated vertically to capture scans of different layers across the
surface of the spinning eye. By knowing the height of the laser
distancers, the angle of the spinning eye, and the measurements of the
distancers, the Eyerez precisely captures the position in physical
space of two points on the surface of the eye which obstruct the
lasers. Once a complete scan is ran, the resultant point cloud is a
highly accurate representation of the shape of the eye.

\dd{There are many benefits to using two lasers. Simultaneous
  measurements create information redundancy which can be used to help
  reduce noise and register the laser positions. Additionally, laser
  distancers, which operate by casting laser light and recording the
  return angle of its reflections, lose accuracy and robustness at
  reflection angles above 80\degree. Two laser systems avoid this
  problem by positioning each laser to specialize in measuring one
  hemisphere of the eye at full accuracy.}

\section{Ideal laser geometry}
\label{sec:laser-geometry}

The lasers used, Keyence LK-G32s, have an optimal measurement distance
of 30\mm\ and can make measurements within $\pm5 \mm$ of that
point. The Eyerez's measurement head thus fixes the lasers in space
atop one another such that they are separated by some height $h$ and
inclined from horizontal trajectories by an angle $\theta$. If the
geometry is constrained so that each laser focuses on the same point
that is 30\mm\ away from the laser face and on a horizontal plane
halfway between the two heads, the distance from the measurement head
to that focal point, $L$, and $h$ are both fully constrained by a
choice of $\theta$:
\begin{align}
  d &= 2\left(30 \sin\theta - 13\cos\theta\right) \\
  L &= 30 \cos\theta + 13 \sin\theta
\end{align}

For optimal laser coverage the angle is chosen as 45\degrees\ which
sets $d = 24.042\mm$ and $L = 30.406\mm$. In general, however, the
angle can range between 25\degrees\ and 90\degrees\ with well-defined
geometry. 

\begin{figure}
  \begin{minipage}{0.5\linewidth}
    \includegraphics[width=\linewidth]{images/laser-geometry}
  \end{minipage}
  \begin{minipage}{0.49\linewidth}
    \caption{\textbf{Ideal laser head geometry.} Here the separation
      height $h$ and the shape of the implemented geometry using a
      45\degree\ angle are clear. The measurement across the head of
      each laser, 26\mm, is also fixed by choice of parts ordered.}
  \end{minipage}
  \label{fig:ideal-laser-geometry}
\end{figure}



%%% Local Variables: 
%%% mode: latex
%%% TeX-master: "../report"
%%% End: 
