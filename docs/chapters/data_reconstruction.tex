
Each measurement from the Eyerez system is a column vector $
\mathbf{x}_{n} \in \mathbb{R}^{4}$ which is indexed in time
discretized by the sampling period, $n : t = T_{s} n$, and includes
the immediate values

\begin{itemize}
\item $d^{u}$, the distance measured from the upper laser
\item $d^{l}$, the distance measured from the lower laser
\item $z$, the linear encoder value relative to some initial point
\item $\theta$, the rotary encoder value
\end{itemize}

Contained within these values are two correlated images of the
measured item encoded linearly into the space of this vector. The
primary goals of data reconstruction are
\begin{inparaenum}[\itshape (a) \upshape]
\item reconstructing the object's image in a sensible, useful
  coordinate space and
\item abusing properties of the data to reduce noise.
\end{inparaenum}

Each of these goals are served through understanding better the
geometry of the measurement system and the effects of possible
perturbations from that geometry while creating techniques to estimate
the exact geometry of the system before and during each measurement.

\section{Overview}
\label{sec:registration-overview}

If $\mathbf{x}_{n}$ is a 4-dimensional column vector representing the
state of the Eyerez system during measurement $n$, the collection of
these $(\mathbf{x}_{n})_{n \leq N}$ contains information on each
laser's image of the object which can together be used to produce a
better overall estimate for the object's shape. To do this, several
steps are involved:

\begin{enumerate}
\item Conversion of each image to a common space for comparison
\item Comparative noise reduction on the two images
\item Combination of images to one complete image
\item Conversion of the denoised, complete image to a space useful for further
  analysis (cylindrical coordinate space)
\end{enumerate}

Each conversion method is a linear operator on $\mathbf{x}_{n}$
dependent on the exact geometry of the device during measurement
(which is currently assumed constant within each measurement
proces). Their construction and analysis is considered in section
\ref{sec:space-registration}. 

The noise reduction is based on the idea during perfect operation
($H_{0}$) the images measured by each laser, denoted $U_{n}$ and
$L_{n}$ across a circular index $\{n\mod N\}$, should be completely
uncorrelated since each laser measures a completely different part of
the object at a given time. However, if the object moves during
measurement then at that point there will be high correlation between
the two images that can be measured and removed. That noise reduction
is not yet implemented.

Finally, the combination step involves finding the time lag $\Delta n$
such that the correlation between the images $U_{n}$ and $L_{n+\Delta
  n}$ is maximized allowing the two to be combined so that the high
and low latitude data is well-covered as would be impossible in a
single-laser configuration.

\section{Space registration}
\label{sec:space-registration}

The observed data vector $\mathbf{x}_{n}$ can be consider the
combination of two other vectors $\mathbf{x}^{u}_{n} = (d^{u} z
\theta)^{T}$ and $\mathbf{x}^{l}_{n} = (d^{l} z \theta)^{T}$ such that
the represent the object as seen by the upper and lower laser
respectively. Space registration thus seeks to find linear operators
$\mathbf{T}_{u}$ and $\mathbf{T}_{l}$ that send each laser's
observation space to a simply interpreted space, such as cylindrical
coordinates. Additionally, assuming invertibility, to create the
``swing-up'' and ``swing-down'' operators $\mathbf{T}_{ul}$ and
$\mathbf{T}_{lu} = \mathbf{T}_{ul}^{-1}$ which allow for easier
comparison of the two images with less dependence on the geometry of
the system.

After these operators are found for ideal geometries they can be used
to better understand how to predict the true operators after
calibration of the system where a small perturbation from the ideal
values is assumed and we are interested in calculating it's value.

\subsection{The basic model: single, orthogonal laser}
\label{sec:basic-model:orthogonal}

The first, simple model is for a single orthogonal laser directed at a
right angle to the rotational axis of the object. For this
configuration, the input vector is slightly simpler than the general
case, $\mathbf{x}_{n} = \left(d, z, \theta\right)$ and the only
required operator is the transformation to a more conveient vector
space. Considering figure \ref{fig:rld=layout} (\textit{left}) it is
easy to construct a matrix which performs this operation

\begin{align}
  \mathbf{T}_{ortho}(L, \Delta z, \Delta \theta)\ \  \mathbf{x} &= \mathbf{y}_{cyl} \\
  \left(
    \begin{array}{cccc}
      -1 & 0 & 0 & L             \\
      0  & 1 & 0 & \Delta z      \\
      0  & 0 & 1 & \Delta \theta \\
      0  & 0 & 0 & 1             \\ 
    \end{array}
  \right)
  \left(
    \begin{array}{c}
      d \\ z \\ \theta \\ 1
    \end{array}
  \right)
  &=
  \left(
    \begin{array}{c}
       L - d \\ z + \Delta z \\ \theta + \Delta\theta \\ 1
    \end{array}
  \right)
  \equiv
  \left(
    \begin{array}{c}
      \hat{r} \\ \hat{z} \\ \hat{\theta} \\ 1
    \end{array}
  \right)
\end{align}

Note the need to include a constant dimension in order to encode the
whole transformation into this array. It is equivalent to a form like
$\mathbf{T}_{ortho} \mathbf{x} + \mathbf{a}$.

It's also important to note that the terms $\Delta z$ and
$\Delta\theta$ are unimportant because for all practical purposes the
images in cylindrical coordinates are the same under constant vertical
shifts and rotations. For that reason, it is useful to simplify
$\mathbf{T}_{ortho}(L) = \mathbf{T}_{ortho}(L, \Delta z,
\Delta\theta)$ with no loss in generality.

\begin{figure}
  \includegraphics[width=\linewidth]{images/rld-layout}
  \caption{\textbf{The laser-measured coordinate space for the single
      laser configuration.} In an orthogonal model, the laser measures
    the distance, $d$, to the measured object. These measurements
    create images of the object in a ``mirrored cylindrical''
    coordinate space (\textit{right}). Given the structural parameter,
    $L$, the measurements can be converted back to images in real
    space.}
  \label{fig:rld-layout}
\end{figure}

\subsubsection{Implications of the simple model}
\label{sec:impl-simple-model}

For this simpler configuration, the form of the transformation
operator is very simple. It is dependent on exactly one unknown value
$L$ which must be estimated during calibration in order to properly
transform the data back into cylindrical coordinates. Additionally, we
know value belief distributions for $L$ from the idealized geometry of
the system and can consider the change in the operator given small
perturbations of the system. In this case, it is easy to see that
$\mathbf{T}_{ortho}(L)$ is not linear in $L$, but still has a simple
decomposition into its ideal and perturbed components
\begin{align}
  \mathbf{T}_{ortho}(L + \delta L) &= \mathbf{P}_{ortho}(\delta
  L)\mathbf{T}_{ortho}(L) \\
  &= \left(I + \left(
    \begin{array}{cccc}
      0 & 0 & 0 & \delta L \\
      0 & 0 & 0 & 0 \\
      0 & 0 & 0 & 0 \\
      0 & 0 & 0 & 0 \\
    \end{array}
  \right)\right) \mathbf{T}_{ortho}(L)
\end{align}

Thus, estimation of the geometric state of the system comes down to
the estimation of $\mathbf{P}_{ortho}$ from which the image from the
laser can be converted to a more usable coordinate space.

\subsection{Two-laser configuration}
\label{sec:two-laser-conf}

The techniques of registration for a two-laser configuration are
fundamentally the same form as the orthogonal laser; however, they are
more complex and involve many more parameters.

First we consider simply the upper laser. The laser impinges on the
measured object at a lattitude higher than the equivalent
orthogonal-laser system would and thus has a much more exotic
geometry. To begin, consider the upper laser geometry from the side
without an impinging object. At this point there is a right triangle
formed between the plane informed by the actuator's value of $z$, the
line of the laser, and a vertical line connecting the two. Calling the
length of the diagonal line $H$, it's vertical component $A$, and
knowing that the horizontal component should be the same $L$ from
figure \ref{fig:ideal-laser-geometry}. Moreover, this geometry is
further constrained by the knowledge that the length of the path of
the laser light should optimally by 30mm, that the height of the laser
body is $26$mm, which for further geometry will be marked $b$, that
the height between the bottom, front corner of the laser body and the
plane measured by $z$ is some height $h/2$, and finally that the laser
is inclined from $0$ degrees in the orthogonal model to $\phi$ degrees
below horizontal.

Now, if an object is impinging on the laser beam, we have a new
triangle geometrically similar to the one formed by $(A, H, L)$ where
the laser beam is shortened from $30$ mm to the measured distance
$d^{u}$. This informs a shortened hypotenuse $H'$, a shortened
vertical component $A'$, and a shortened horizontal component which we
are directly interested in called $d = L - r$. From the geometry of
similar triangles we also know
\begin{align}
  \frac{A'}{A} = \frac{H'}{H} = \frac{d}{L}
\end{align}

and from the geometry of the system we know
\begin{align}
  L = H \cos\phi
\end{align}

from which we can create the operator to transform from the upper
coordinate system to the cylindrical coordinate system via
\begin{align}
  \mathbf{T}_{u}(\phi, \Delta \theta)\ \  \mathbf{x} &= \mathbf{y}_{cyl} \\
  \left(
    \begin{array}{cccc}
      \sin\phi  & 0 & 0 & 30 \sin\phi   \\
      -\cos\phi & 1 & 0 & 30 \cos\phi   \\
      0         & 0 & 1 & \Delta \theta \\
      0         & 0 & 0 & 1             \\ 
    \end{array}
  \right)
  \left(
    \begin{array}{c}
      d \\ z \\ \theta \\ 1
    \end{array}
  \right)
  &=
  \left(
    \begin{array}{c}
      \hat{r} \\ \hat{z} \\ \hat{\theta} \\ 1
    \end{array}
  \right)  
\end{align}

which may be noted has a strong similarity to a 2-dimensional rotation
matrix. Also note that dependency on L vanishes but we introduce a new
term $\Delta\theta$ which encompasses mismatch between the upper and
lower laser's rotary position. This leaves 3 free parameters in this
transform.

The corresponding transform, $\mathbf{T}_{l}(\phi)$ differs from
$\mathbf{T}_{u}(\phi, \Delta\theta)$ in that $\Delta\theta$ is set to
0 by definition and $\mathbf{T}_{l}^{2,2} = -\mathbf{T}_{u}^{2,2}$
representing that the lower laser measures points below the plane of
$z$ instead of above.

\subsubsection{Perturbations in the two-laser configuration}
\label{sec:pert-two-laser}

With definitions for $\mathbf{T}_{u}$ and $\mathbf{T}_{l}$ we can
consider the effects of small pertubations from the ideal operators
$\mathbf{T}_{u}^{*}$ and $\mathbf{T}_{l}^{*}$. Here, starting with the
upper laser, since $\Delta\theta$ is already an expected perturbation
from 0, we simply substitute $\phi = \phi^{*} + \delta\phi_{u}$ and
use derivative linearization to approximate a linear form for it.
\begin{align}
  \mathbf{T}_{u}(\phi^{*}+\delta\phi) &\approx
  \mathbf{T}_{u}(\phi^{*}) + \delta\phi
  \mathbf{D}\{\mathbf{T}_{u}(\phi)\}|_{\phi = \phi^{*}} \\
  &\approx
  \left(
    \begin{array}{cccc}
      \delta\phi \cos\phi^{*}+\text{Sin}[\phi^{*} ] & 0 & 0 & 30 (\delta
      \phi \cos\phi^{*} + \sin\phi^{*}) \\
      -\cos\phi^{*} + \delta\phi \sin\phi^{*} & 1 & 0 & 30
      (\cos\phi^{*} - \delta\phi \sin\phi^{*}) \\
      0 & 0 & 1 & \Delta \theta  \\
      0 & 0 & 0 & 1
    \end{array}
  \right)
\end{align}

which can be factored as in the single laser form into an ideal and
perturbed forms
\begin{align}
  \mathbf{T}_{u}(\phi + \delta\phi) = \mathbf{P}_{u}(\phi, \delta\phi)\mathbf{T}_{u}^{*}(\phi)
\end{align}

where $\mathbf{T}_{u}^{*} = \mathbf{T}_{u}(\phi, \Delta\theta)$ and
\begin{align}
  \mathbf{P}_{u} = \left(
    \begin{array}{cccc}
      1 + \delta\phi \cot\phi & 0 & 0 & 0 \\
      \delta\phi & 1 & 0 & -60 \delta\phi \sin\phi \\
      0 & 0 & 1 & 0 \\
      0 & 0 & 0 & 1 \\
    \end{array}
  \right)
\end{align}

which is conveniently the exact same for the lower laser
$\mathbf{P}_{l} = \mathbf{P}_{u}$.

\subsubsection{Swingup and swingdown}
\label{sec:swingup-swingdown}

The swingup matrix is the matrix which converts the lower laser
coordinates to the upper laser coordinates. It's defined as
\begin{align}
  \mathbf{T}_{\uparrow} &= \mathbf{T}_{u}^{-1}\mathbf{T}_{l} \\
  &= (\mathbf{P}_{u}\mathbf{T}_{u}^{*})^{-1}
  (\mathbf{P}_{l}\mathbf{T}_{l}^{*}) \\
  &= \mathbf{T}_{u}^{*, -1}\mathbf{P}_{u}^{-1}
  (\mathbf{P}_{l}\mathbf{T}_{l}^{*}) \\
\end{align}

Which motivates the calculation of $\mathbf{P}_{\uparrow} =
\mathbf{P}_{u}^{-1}\mathbf{P}_{l}$ as
\begin{align}
  \mathbf{P}_{\uparrow} = 
  \left(
  \begin{array}{cccc}
    \frac{1 + \delta\phi_{l} \cot\phi}{1 + \delta\phi_{u} \cot\phi} &
    0 & 0 & 0 \\
    \frac{\delta\phi_{l} - \delta\phi_{u}}{1 + \delta\phi_{u}
      \cot\phi} & 1 & 0 & 60
    \left(\delta\phi_{u}-\delta\phi_{l}\right)\sin\phi \\
    0 & 0 & 1 & 0 \\
    0 & 0 & 0 & 1 \\
  \end{array}
  \right)
\end{align}

With the swingdown matrix $\mathbf{P}_{\downarrow}$ similarly defined
being the obvious form,
\begin{align}
  \mathbf{P}_{\uparrow} = 
  \left(
  \begin{array}{cccc}
    \frac{1 + \delta\phi_{u} \cot\phi}{1 + \delta\phi_{l} \cot\phi} &
    0 & 0 & 0 \\
    \frac{\delta\phi_{u} - \delta\phi_{l}}{1 + \delta\phi_{l}
      \cot\phi} & 1 & 0 & 60
    \left(\delta\phi_{l}-\delta\phi_{u}\right)\sin\phi \\
    0 & 0 & 1 & 0 \\
    0 & 0 & 0 & 1 \\
  \end{array}
  \right)
\end{align}

\subsection{Limitations}
\label{sec:limitations}

The above section suggests methods for formalizing the geometric
properties of the two-laser system so that measurements from each
laser can be interpreted in any space while dealing with small
misalignment errors in a linear fashion. These concerns are necessary
in the creation of estimation systems for those small misalignments
(calibration systems) and also for the ``correction'' of measured data
to a useful space. 

\dd{The current approach is worrisome though in that it appears to
  have fewer degrees of freedom than should be necessary. For
  instance, if the distance between the lasers changes then the
  transformation matrices do not take that translation into
  account. It likely has to to do with the vanishing of the L term due
  to the constraints in ideal geometry though I'm not clear how to fix
  it at the moment. It is however much simpler at the moment and very
  composable which is useful for further analysis.}

%%% Local Variables: 
%%% mode: latex
%%% TeX-master: "../report.tex"
%%% End: 
