
\section{Space registration}
\label{sec:space-registration}

The data produced by each laser in the Eyerez system must be
registered to real space coordinates before being useful as a
representation of the part scanned. This registration is further
hampered by the fact that the exact position of each laser in space is
not completely known and must be computed. Finally, the parameter $L$
from Figure \ref{fig:ideal-laser-geometry}, must be measured
independently using a part with a known configuration.

\subsection{The basic model: single, orthogonal laser}
\label{sec:basic-model:orthogonal}

A single laser that is scanning at a right angle from the rotational
axis of the eye produces data corresponding to the measured distance,
the current $z$-position, and the current $\theta$-position. These
triples $(d_{laser}, z_{laser}, \theta_{laser})$ exist in an inverted
polar coordinate system that has been mirrored across the central
axis. Since the laser head is at a right angle to the rotation axis,
however, conversion to real polar space $(r, z, \theta)$ is very easy
given knowledge of the parameter $L$.

\begin{align}
  r &= L - d_{laser} \\
  z &= z_{laser} \\
  \theta &= \theta_{laser}
\end{align}

\begin{figure}
  \includegraphics[width=\linewidth]{images/rld-layout}
  \caption{\textbf{The laser-measured coordinate space.} In an
    orthogonal model, the laser measures the distance, $d$, to the
    measured object. These measurements create images of the object in
    a mirrored polar coordinate space (\textit{right}). Given the
    structural parameter, $L$, the measurements can be converted back
    to images in real space.}
  \label{fig:ideal-laser-geometry}
\end{figure}

\subsection{Two-laser registration}
\label{sec:two-laser-registr}

While the basic model suggests the importance of the structural
parameter $L$, it fails to relate the full complexity of two-laser
geometry. In reality, each laser is translated and rotated



%%% Local Variables: 
%%% mode: latex
%%% TeX-master: "../report.tex"
%%% End: 
